\documentclass[12pt,a4paper]{report}
\usepackage[utf-8]{inputenc}
\usepackage[margin=2.5cm]{geometry}
\usepackage{graphicx}
\usepackage{hyperref}
\usepackage{amsmath}
\usepackage{amssymb}
\usepackage{listings}
\usepackage{xcolor}
\usepackage{float}
\usepackage{caption}
\usepackage{subcaption}
\usepackage{booktabs}
\usepackage{array}
\usepackage{setspace}

% Listings configuration
\lstset{
    language=Python,
    basicstyle=\ttfamily\small,
    keywordstyle=\color{blue},
    commentstyle=\color{gray},
    stringstyle=\color{red},
    breaklines=true,
    postbreak=\mbox{\textcolor{red}{$\hookrightarrow$}\space},
    frame=single,
    backgroundcolor=\color{white},
    numbers=left,
    numbersep=5pt,
    showstringspaces=false
}

% Set line spacing
\onehalfspacing

% Title page
\title{\textbf{System Architecture and Analysis Report}}
\author{Author Name}
\date{\today}

\begin{document}

\maketitle

\tableofcontents
\newpage

% ============================================================================
% CHAPTER 1: INTRODUCTION
% ============================================================================
\chapter{Introduction}

\section{Purpose of the System}

Provide a comprehensive description of the overall purpose and objectives of the system you are studying. Explain the problem the system aims to solve and the value it provides.

\section{Company and Historical Context}

Describe the organization or research group that developed the system. Provide historical background on its development, including key milestones, versions, and evolution over time.

\section{Main Features of the System and Novelty}

List and explain the key features that distinguish this system from others in its domain. Highlight the novel contributions and innovations introduced by this system, including:
\begin{itemize}
    \item Technical innovations
    \item Performance improvements
    \item Usability enhancements
    \item Scalability features
\end{itemize}

\section{System Architecture Overview}

Provide a high-level overview of how the system is organized and structured. This section introduces the architectural approach that will be detailed in Chapter 2.

% ============================================================================
% CHAPTER 2: SYSTEM ARCHITECTURE
% ============================================================================
\chapter{System Architecture}

\section{Architecture Diagram}

\begin{figure}[H]
    \centering
    \includegraphics[width=0.9\textwidth]{architecture_diagram.png}
    \caption{Complete system architecture showing all major components and their relationships}
    \label{fig:architecture}
\end{figure}

\section{Component Description}

This section provides detailed descriptions of each major component in the system.

\subsection{Component 1: [Name]}

Describe the first major component, including:
\begin{itemize}
    \item Primary responsibilities
    \item Interfaces and APIs
    \item Key algorithms or techniques
    \item Implementation details
\end{itemize}

\subsection{Component 2: [Name]}

Describe the second major component following the same structure as above.

\subsection{Component 3: [Name]}

Describe additional components as needed.

\section{Specific Architectural Features}

Describe notable architectural design decisions and patterns employed:

\subsection{[Feature Name 1]}

Explain the architectural feature, why it was chosen, and how it addresses specific design challenges.

\subsection{[Feature Name 2]}

Provide detailed explanation of additional key architectural features.

\subsection{Data Flow and Communication}

Describe how data flows through the system and how components communicate with each other.

% ============================================================================
% CHAPTER 3: SPECIFIC CONCEPT STUDIED IN SEMINAR
% ============================================================================
\chapter{Specific Concept Studied in Seminar}

\section{Concept Definition and Background}

Define and introduce the specific technical concept or methodology that was the focus of your seminar analysis. Provide relevant background and context.

\section{Technical Details}

Provide an in-depth technical explanation of the concept, including:
\begin{itemize}
    \item Theoretical foundations
    \item Mathematical formulations (if applicable)
    \item Algorithms or procedures
    \item Known limitations and challenges
\end{itemize}

\subsection{Subsection Example}

Include formulas, equations, or technical details:

\[
    f(x) = \frac{1}{2\pi\sigma^2} \exp\left(-\frac{x^2}{2\sigma^2}\right)
\]

\section{Implementation in the System}

Explain how this specific concept is implemented and utilized within the studied system. Provide code examples if relevant:

\begin{lstlisting}
# Example code snippet
def example_function(parameter):
    """
    Description of function
    """
    result = parameter * 2
    return result
\end{lstlisting}

\section{Significance and Impact}

Discuss the importance of this concept within the broader context of the system and its implications.

% ============================================================================
% CHAPTER 4: TEST APPLICATION
% ============================================================================
\chapter{Test Application}

\section{Purpose and Design}

\subsection{Objectives}

State the primary objectives of developing and running the test application. What specific aspects of the system were you testing or validating?

\subsection{Test Design}

Describe the design methodology, including:
\begin{itemize}
    \item Test scenarios and use cases
    \item Performance metrics measured
    \item Experimental setup and configuration
    \item Data sets used in testing
\end{itemize}

\subsection{Test Environment}

Specify the testing environment details:
\begin{itemize}
    \item Hardware specifications
    \item Software dependencies and versions
    \item Configuration parameters
    \item Any constraints or limitations
\end{itemize}

\section{Implementation of Test Application}

\subsection{Architecture and Components}

Describe the structure of your test application, including major components and their interactions.

\subsection{Implementation Details}

Provide implementation specifics:

\begin{lstlisting}
# Test application code example
class TestApplication:
    def __init__(self, config):
        self.config = config
        self.results = []
    
    def run_test(self, test_case):
        """Run a single test case"""
        # Implementation details
        pass
\end{lstlisting}

\subsection{Experimental Procedures}

Detail the step-by-step procedures followed during testing:
\begin{enumerate}
    \item Initialize the system with specific parameters
    \item Execute test cases in a defined sequence
    \item Record measurements and observations
    \item Repeat for different configurations or inputs
\end{enumerate}

\section{Experiments and Results}

Document each experiment conducted:

\subsection{Experiment 1: [Name]}

\subsubsection{Setup and Parameters}

Describe the specific configuration for this experiment.

\subsubsection{Execution}

Explain how the experiment was executed.

\subsubsection{Observations}

Document observations and measurements collected.

\subsection{Experiment 2: [Name]}

Follow the same structure for additional experiments.

% ============================================================================
% CHAPTER 5: RESULTS AND MAIN TAKEAWAYS
% ============================================================================
\chapter{Results and Main Takeaways}

\section{Experimental Results}

\subsection{Quantitative Results}

Present numerical results in tables and figures:

\begin{table}[H]
    \centering
    \caption{Results comparison across different configurations}
    \label{tab:results}
    \begin{tabular}{lccccc}
        \toprule
        \textbf{Configuration} & \textbf{Metric 1} & \textbf{Metric 2} & \textbf{Metric 3} & \textbf{Time (ms)} \\
        \midrule
        Baseline & 95.2\% & 1.2 & 0.87 & 145 \\
        Configuration A & 96.5\% & 1.5 & 0.92 & 152 \\
        Configuration B & 97.1\% & 1.8 & 0.95 & 168 \\
        \bottomrule
    \end{tabular}
\end{table}

\subsection{Qualitative Results}

Summarize qualitative observations and findings from the experiments.

\begin{figure}[H]
    \centering
    \includegraphics[width=0.8\textwidth]{results_graph.png}
    \caption{Performance comparison across different experimental configurations}
    \label{fig:results}
\end{figure}

\section{Key Findings}

Present the most important findings from your experiments:
\begin{itemize}
    \item Finding 1: Description and implications
    \item Finding 2: Description and implications
    \item Finding 3: Description and implications
\end{itemize}

\section{Main Takeaways}

Summarize the primary conclusions drawn from your experimental work:

\subsection{Performance Characteristics}

Discuss what the results reveal about the system's performance.

\subsection{System Validation}

Explain what your experiments validated or challenged about the system design.

\subsection{Practical Implications}

Discuss the practical significance of your findings.

% ============================================================================
% CHAPTER 6: CONCLUSIONS
% ============================================================================
\chapter{Conclusions}

\section{Comments on the Studied System}

Provide your critical analysis and evaluation of the system:
\begin{itemize}
    \item Strengths and advantages
    \item Weaknesses and limitations
    \item Areas of effectiveness
    \item Potential improvements
\end{itemize}

\subsection{System Strengths}

Elaborate on what the system does well and its competitive advantages.

\subsection{System Limitations}

Discuss limitations, scalability concerns, or design trade-offs.

\subsection{Architectural Decisions}

Evaluate the architectural choices made and their effectiveness.

\section{Remarks on the Test Application}

\subsection{Test Design Effectiveness}

Evaluate how well your test application measured what it was intended to measure.

\subsection{Experiment Validity}

Discuss the validity of your experimental approach and any limitations.

\subsection{Reproducibility}

Comment on the reproducibility of your experiments and any factors that might affect this.

\section{Lessons Learned and Experiences}

Share key lessons and insights gained through this study:

\subsection{Technical Insights}

Document technical lessons learned about the system and the specific concept studied.

\subsection{Methodological Insights}

Share insights about the research and testing methodology employed.

\subsection{Future Directions}

Suggest areas for future research or improvements based on your findings.

\subsection{Personal Reflections}

Share any notable experiences or reflections from conducting this analysis.

\section{Final Summary}

Provide a concise summary of your overall conclusions and the significance of your work.

% ============================================================================
% REFERENCES
% ============================================================================
\begin{thebibliography}{99}

\bibitem{ref1}
Author, A., Author, B., \textit{Title of Paper or Book}. Publisher, Year.

\bibitem{ref2}
Author, C., \textit{Title of Paper}. Journal Name, Volume(Issue), Pages, Year.

\bibitem{ref3}
Author, D., et al., \textit{Title of Conference Paper}. In Proceedings of Conference Name, Year.

\bibitem{ref4}
Organization, \textit{Title of Technical Report}. Technical Report, Year.

\bibitem{ref5}
Author, E., \textit{Web Resource Title}. URL, Accessed: Date.

\end{thebibliography}

\end{document}